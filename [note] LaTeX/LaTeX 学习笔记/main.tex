%\documentclass[11pt, a4, twocolumn]{article}
\documentclass[11pt, a4paper]{article}

\usepackage{ctex} % 中文支持
\usepackage[
    left=2.3cm, right=2.3cm, top=1.54cm, bottom=1.5cm
    %,includeheadfoot, headheight = 2.3cm, footskip = 2.9 cm
    ]{geometry}  % 设置文档格式
\usepackage{graphicx}   % 用于导入图片
\usepackage{verbatim}   % 批量注释,使用\begin{comment}环境

\usepackage{wrapfig}    % 可以被文字包围的图片
\usepackage{lipsum}     % 生成随机话语,使用\lipsum[1-2]环境
%\usepackage{cuted}     % 双栏跨行支持,使用\begin{strip}环境
%%\stripsep -3pt plus 3pt minus 2pt
%\usepackage{amsmath,esint} % 输出数学公式
\usepackage{fancyhdr}
\usepackage[dvipsnames]{xcolor}
    \pagestyle{fancy}
    \fancyhf{}
    \rhead{\LaTeX{} 从入门到魔怔}
    \lhead{张家铭}
    \rfoot {\thepage}
    \lfoot{2022}

\usepackage{listings}   % 代码高亮和格式化展示
    \usepackage{xcolor}
    \definecolor{codegreen}{rgb}{0,0.6,0}
    \definecolor{codegray}{rgb}{0.5,0.5,0.5}
    \definecolor{codepurple}{rgb}{0.58,0,0.82}
    \definecolor{backcolour}{rgb}{0.95,0.95,0.92}
    \lstdefinestyle{mystyle}{
        backgroundcolor=\color{backcolour},   
        commentstyle=\color{codegreen},
        keywordstyle=\color{magenta},
        numberstyle=\tiny\color{codegray},
        stringstyle=\color{codepurple},
        basicstyle=\ttfamily\footnotesize,
        breakatwhitespace=false,         
        breaklines=true,                 
        captionpos=b,                    
        keepspaces=true,                 
        numbers=left,                    
        numbersep=7pt,                  
        showspaces=false,                
        showstringspaces=false,
        showtabs=false,                  
        tabsize=8,
        frameround=fftt,
        frame=tBLr,
        emph={printf},emphstyle={\underbar}, % 强调printf,在它下面加下划线
        aboveskip = 3pt, %与代码环境上一行的垂直间距为5pt
        belowskip = 3pt %与代码环境下一行的垂直间距为4pt
    }
    \lstset{style=mystyle}

\begin{comment}
\fancyhead[C]{
    \noindent
    \fcolorbox{SpringGreen}{SpringGreen}{%
        \begin{minipage}[c][2.3cm][c]{0.025\headwidth}
            \mbox{}    
        \end{minipage}
        \begin{minipage}[c][2.3cm][c]{0.5\headwidth}
            \LARGE{
                \textcolor{darkgray}{This is the Title of a Very Important Report}}\\\large{\textcolor{gray}{Author}
                }
        \end{minipage}
        \begin{minipage}[c][2.3cm][c]{0.45\headwidth}%
            \hfill%
            \includegraphics[height=0.9\headheight]{fig/head.jpeg}    
        \end{minipage}
        \begin{minipage}[c][2.3cm][c]{0.025\headwidth}
             \mbox{}
        \end{minipage}
    }
}
\renewcommand{\headrulewidth}{2pt}
\renewcommand{\footrulewidth}{1pt}
\fancyhfoffset[L]{2.3cm}
\fancyhfoffset[R]{2.3cm}
\end{comment}

\bibliographystyle{plain}
    \nocite{*}

\title{\LaTeX{} 从入门到魔怔}
\begin{document}
	% 在页面的顶部生成一个标题
	\maketitle
	\tableofcontents
	\section{\LaTeX{} 简介}
		\LaTeX{} 是一个使用高级标记语言的排版系统。\LaTeX{} 使用一组丰富的命令来指定文档的组成部分(标题、章节、粗体字、斜体字、图表等等),并让\LaTeX{} 来处理格式问题。当然,你还可以对文件进行个性化处理,进一步指定文件的格式和功能。
		
		一个\LaTeX{} 文档(*.tex文件)被分成两个主要部分:导言区(preamble)和文件的主体(main body)。导言区包含了所有的全局规格和格式设置,如纸张大小、字体大小、是什么类型的文件(文章、书籍、报告、信件\dots)。它也是我们包含 "包 "的地方,它允许访问宏库以增加\LaTeX{} 的功能。然后是文件的主体,通常被分为不同的部分、章节或小节。
		
		导言区从这个命令开始:
		\begin{lstlisting}
    \documentclass[ ]{ }\end{lstlisting}
		大括号允许你指定文件类型(或 "类别")--标准的类型是article、book、report和letter。写完了导言区,主文件的开头和结尾是:
		\begin{lstlisting}
    \begin{document} ... \end{document}\end{lstlisting}
		在\textbackslash begin\{document\} 以前的命令,将被视为处于导言区中的命令。
		
		\subsection{文档类别}
			\LaTeX{} 的标准分段命令是:\textbackslash part\{\}, \textbackslash section\{\}, \textbackslash subsection\{\}和\textbackslash paragraph\{\}。这些都是三个标准文档类别所共有的,而report和book还有一个更高的分段命令\textbackslash chapter\{\},article还有一个更低的分段命令\textbackslash subsection\{\}。
			
			还有一些其他区别,诸如:
			\begin{itemize}
				\item[1] book和report中的每一章都有图表和其他图表以及方程式的编号,而文章中的编号则是全面的;
				\item[2] 书籍和报告类的文件以标题页(如果指定的话)作为独立的一页开始,而文章则没有;
				\item[3] \textbackslash abstract\{\}在book类中是不可用的
			\end{itemize}
			\textbackslash documentclass[ ]\{ \}中的方括号为可选参数,有如下几类:
			\begin{enumerate}
				\item[纸张尺寸] 标准选项包括:a4纸、a5纸、b5纸、letterpaper、executivepaper和legalpaper。默认的页面大小是US letter,类似于A4。
				\item[字体大小] 如12pt。对于标准文档类别,可以接受10pt(默认)、11pt和12pt。进一步的字体大小规范可以在文档正文中实现;
				\item[单面双面] 默认情况下,书籍是双面的,而文章和报告是单面的;
				\item[草稿模式] 不编排图片,允许更快的编排速度。
			\end{enumerate}
			
			这些只是设置了全局的格式,在文件的主体部分,你可以在需要时对某些选项进行局部修改。在指定了文档类别和所需的选项后,导言部分还允许我们调用“包”,这样我们就可以访问\LaTeX{} 中特定功能的宏。包含这些的命令是:
			\begin{lstlisting}
    \usepackage[]{}\end{lstlisting}
			流行的软件包包括geometry、graphicx、amsmath、xcolor,以及其他许多软件包。包的资源库及其文档可以在www.ctan.org。表\ref{table:pack}给出了最流行的软件包和它们的用途的概述。
			
			\begin{table}[!h]
			    \centering
				\begin{tabular}{|l|l|}
				\hline
						包名称              & 作用\\ \hline
						graphicx           & 在文档中包含图片的必要条件。提供includegraphics命令\\
						babel              & 改变文件的默认语言(对正确的连字符和换行符有用)\\
						amsmath, amssymb   & 提供了一套全面的命令来格式化数学语句以及符号\\
						hyperref           & \LaTeX{} 中的超链接管理,允许链接到文本中的
						                     图、表和其他项目。\\
						geometry           & 允许根据需要定义页面设置,如控制边距和文本宽度等功能\\
						microtype          & 控制字距和其他字体设置\\                                                                                       
						booktabs, multicol,\\ tabularx, tabulary 
						                   & 高级表格格式化\\
                        tikz, pgfplots     & 绘图和制图软件包(第三章)\\
						fancyhdr           & 构建页眉和其他高级页面格式化\\                                                                                       
                        biblatex, natbib   & 参考文献管理(第二章)\\ \hline                                                                          
				\end{tabular}
				\caption{\LaTeX{} 中常用的包}
				\label{table:pack}
			\end{table}
			
			通常需要进行的第一项修改是调整页边距和行距。要做到这一点,可以使用geometry包。它不仅可以用各种单位来规定页边距,还可以规定许多其他参数。边距可以用:
			\begin{lstlisting}
    \usepackage[margin=1.5cm]{geometry}\end{lstlisting}
			来设置,也可以单独指定每个边距,用
			\begin{lstlisting}
    \usepackage[left=2cm, right=2cm, top=1.54cm, bottom=1cm]{geometry}\end{lstlisting}
			改变行距的一个简单方法是使用setspace包。通过在导言区中设置usepackage[...]{setspace},我们可以设置[doublespacing]来获得一个双倍行距的文件,或使用:
			\begin{lstlisting}
    \usepackage[onehalfspacing]{setspace}\end{lstlisting}

		\subsection{基础环境}
			\subsubsection{图片}
				在\LaTeX{} 中导入 graphicx 包,我们可以使用\textbackslash includegraphics[ ]\{\}命令引入一个图片,并且可以在方括号位置直接定义图片的尺寸,如:[scale=1.3];同时也可以间接设置图片的大小:
				\begin{lstlisting}
    \includegraphics[width = 0.8 \textwidth]{figure.png}\end{lstlisting}
				
				上面的例子是将图片调整一个为:页宽80\%的图片并引入。如果想同时指定只需要设置为:[width = 2cm, height = 5 cm],想要旋转一定角度也可以设置:[angle=90]。
				
				在大括号中,我们必须传递图片的名称(路径名)。文件扩展名可以省略。\LaTeX{} 搜索图片的默认位置可以在导言中设置为全局变量:
				\begin{lstlisting}
    \graphicspath{{./figures}{./additionalFigures}}\end{lstlisting}
				通过将 \textbackslash includegraphics[ ]\{\}命令放在 \textbackslash begin\{figure\}...\textbackslash end\{figure\}环境中,我们可以指定其相对于页面的位置,并给该图一个标题和标签。我们可以通过使用 \textbackslash begin\{figure\}[h] 来告诉\LaTeX{} ,让它在文本中的位置与源代码中的位置大致相同。如果我们想把图表放在页面的顶部或底部,我们可以用 t 或 b 代替 h;使用 p 可以把它放在自己的页面上,而添加一个 !可以覆盖\LaTeX{} 默认的定位参数。\LaTeX{} 中图表的默认对齐方式是在页面的左边。由于图像经常需要居中,在图的环境中放置命令\textbackslash centering可以实现这一目的。
				
				标题的位置是在图的上方还是下方,只取决于\textbackslash caption\{...\}命令是在 \textbackslash includegraphics[]\{\}命令的上方还是下方。标签很有用,因为我们可以用它们来指代文本中的一个图(或表格、方程式等),而\LaTeX{} 会自动用正确的序号来代替它。
				\begin{figure}[h]
				    \centering
				    \includegraphics[width = 0.2\textwidth]{fig/head}
				    \caption{我的微信头像}
				    \label{fig:MyHead}
				\end{figure}
				有时候我们可能希望我们的文字包裹着图片,我们则需要引入 wrapfig 包。同时,我们通过使用\textbackslash begin\{wrapfigure\}\{alignment\}\{width\} ... \textbackslash end\{wrapfigure\} 环境。对于对齐方式,r 对应右对齐、l 对应左对齐,而对于宽度则和常规图形环境相同。实际的图形本身也像以前一样被插入。
								
%------------------------------------------------------------------------------------------
%
%------------------------------------------------------------------------------------------	
				\lipsum[1]
				\begin{wrapfigure}{r}{0.2\textwidth}
				    \centering
				    \includegraphics[width = 0.2\textwidth]{fig/head}
				    \caption{右对齐}
				    \label{fig:LeftHead}
				\end{wrapfigure}
				
				(测试文字)\lipsum[1-3]
				
				\begin{wrapfigure}{l}{0.2\textwidth}
				    \centering
				    \includegraphics[width = 0.2\textwidth]{fig/head}
				    \caption{左对齐}
				    \label{fig:LeftHead}
				\end{wrapfigure}
				
				请注意,图的宽度不是相对于图框而言的,而是相对于文本宽度而言的,因此我们将它们的宽度设置为相同,这样图就可以填满图框的空间。
				
				(测试文字)\lipsum[30]
%------------------------------------------------------------------------------------------
%
%------------------------------------------------------------------------------------------

			\subsubsection{表格}
			     \LaTeX{} 提供了许多软件包和工具来生成表格。两个常用的环境即 table 和 tabular,前者提供了表格的定位、对齐、标签和标题,后者则是表格本身。需要注意的是,table 环境不是必需的,我们可以简单地用 \textbackslash begin\{tabular\} 和\textbackslash end\{tabular\} 插入一个表格,但是这将让我们能够调整的参数变少。
			     
			     \begin{table}[!h]
			     \centering
			         \begin{tabular}{ l l l }
			             cell 11 & cell 12 & cell 13 \\
			             cell 21 & cell 22 & cell 23 \\
			         \end{tabular}
			     \caption{Table caption}
			     \label{tab:myspstable}
			     \end{table}


                如同在 Figure 环境中一样,方括号中的位置变量允许我们将表格放置在各种地方。\textbackslash centering 使得表格居中放置。在 Tabular 环境中,右侧的 \{\} 表示表格内部元素的对齐方式:
                \begin{table}[!h]
                \centering
                    \begin{tabular}{|c|c|c| p{3cm} |c|}
                    \hline
                        l & r & c & p\{3 cm\} & |\\ \hline
                        左对齐 & 右对齐 & 居中放置 & 列宽 3 cm & 边界\\ \hline    
                    
                    \end{tabular}
                    \caption{tabular 命令的参数列表}
                    \label{tab:myspstable2}
                \end{table}

			    在每一行下部画一个横线可以使用, \textbackslash hline 命令。
	\section{进阶操作}
	
	本章涉及高级格式化选项,如参考文献管理、页面设置和字体管理。此外,还介绍了新命令的创建以及与其他格式和编程语言的整合。本章最后对交叉引用、超链接和其他工具进行了一些介绍。
	
	   \subsection{参考管理系统}
	   
	       对于任何需要大量参考文献的大型作品和文件来说,最好先生成一个包含所有参考文献的.bib文件。这个文件必须按照特定模式进行格式化,然后可以用来在\LaTeX{} 中引用。
	       
	       要生成这个.bib文件,有两种选择:
	       \begin{itemize}
	           \item 使用允许导出.bib文件的参考文献管理程序(由于\LaTeX{} 的广泛采用,流行的参考文献管理程序如Zotero、Mendeley或EndNote都有这种功能);
	           \item 也可以手动创建一个.bib文件,这只需要一个简单的文本文件,其中包含一个所需的.bib格式的参考文献列表。
	       \end{itemize}
	       
	       在.bib文件中,类型用 @ 表示(例如:@book),然后在大括号内给出条目,字段之间用逗号分隔;
	       
	       \begin{itemize}
	           \item 第一个字段是关键词(在引用时使用);
	           \item 作者姓名用 <姓,名> 来表示,并用 "and" 来分隔;
	           \item 当作者字段中出现机构或组织时,最好用一组额外的大括号将其括起来,以避免自动缩短第一个 “名字”。
	       \end{itemize}
	       
	       BIBTeX 用样式(style)来管理参考文献的写法。BIBTeX 提供了几个预定义的样式,如 plain, unsrt, alpha 等。如果使用期刊模板的话,可能会提供自用的样式。 样式文件以 .bst 为扩展名。使用样式文件的方法是在源代码内(一般在导言区)使用 \textbackslash bibliographystyle 命令:
	       \begin{lstlisting}
    \bibliographystyle{ < < bib-name > > }\end{lstlisting}

	       这里 < bst-name > 为 .bst 样式文件的名称,不要带 .bst 扩展名。当准备好了 .bib 数据库并且设置好样式文件后,在正文中使用 \textbackslash cite 命令去引用一个参考文献\cite{benson2013cme},若希望没在正文中被引用的文献也出现在结尾的列表中,请使用 \textbackslash nocite\{*\} 命令。最后,在需要列出参考文献的位置,使用 \textbackslash bibliography 命令:

	       \begin{lstlisting}
    \bibliography{ < bib-name > }\end{lstlisting}

	   \subsection{页面设置}
	       在\LaTeX{} 中,页面的大小是在一开始就用 \textbackslash documentclass 命令设置的(例如,用 \textbackslash documentclass[a4paper]\{article\}),而更精细的调整。有两个包在这方面特别有用,即 geometry 包和 fancyhdr 包。
	       
	       \begin{table}[h]
	           \centering
	           \begin{tabular}{|c|c|}
	                   \hline
	                   参数 & 作用\\
	                   \textbackslash pageheight, \textbackslash pagewidth   
	                                             & 文档页面高度、宽度 \\ \hline
	                   \textbackslash textwidth, \textbackslash textheight   
	                                             & 文本区域的宽度、高度\\ \hline
	                   \textbackslash linewidth                & 某一行宽度\\ \hline
	                   \textbackslash baselineskip             & 段内行间距 \\ \hline
	                   \textbackslash parskip                  & 自然段间距\\ \hline
	           \end{tabular}
	           \caption{\LaTeX{} 可用空间参数}
	       \end{table}
	       
	       有了这些参数作为参考系,我们可以很容易地定义一些对象的尺寸,比如我们之前做过的,我们可以使用 \textbackslash includegraphics[width=0.5\textbackslash linewidth]\{\} 来定义一个 50\% 行宽度的图片。另一方面,我们也可以使用确切的数值以及单位:cm, mm, pt, sp, in, a和 em/ex (分别代表 ‘M’ 和 ‘x’).
	       
	       \subsubsection{留边与页面尺寸}
	       \subsubsection{颜色}
	       \subsubsection{页眉和页脚}
	           包 fancyhdr 可以帮助我们实现精美的页眉和页脚,它允许我们对页眉和页脚进行完全定制。页眉和页脚的具体参数要在导言区进行设置,最常见的是使用命令:
	           \begin{table}[h]
	               \centering
	               \begin{tabular}{|c|c|}
	                       参数 & 作用\\
	                       \textbackslash rhead\{\} & 右上文字\\
	                       \textbackslash lhead\{\} & 左上文字\\
	                       \textbackslash chead\{\} & 中上文字\\
	                       \textbackslash rfoot\{\} & 右下角\\
	                       \textbackslash lfoot\{\} & 左下角\\
	                       \textbackslash cfoot\{\} & 中下部\\
	               \end{tabular}
	               \caption{fancyhdr 常用字段}
	               \label{table:fancy}
	           \end{table}
	           一些特殊的字段,例如,\textbackslash thepage将生成页码,类似的还有\textbackslash thechapter和\textbackslash thesection,而\textbackslash leftmark和\textbackslash rightmark则分别生成当前章节的名称和编号(报告或书籍/文章),以及章节/分节(报告或书籍/文章)。
	           \begin{lstlisting}
    \renewcommand{\headrulewidth}{... pt}\end{lstlisting}
	           可以用上述的命令来规定页眉(foot, 页脚)与正文之间的分隔线的粗细。为任何一行设置 0pt,就可以删除该行。
	           
	           在下面的例子中,我们将尝试为我们页面的顶部和底部生成一个横幅,以帮助示范其中的一些命令。因此,在序言中,我们将首先消除页眉、页脚和文本之间的线条,并使页眉也覆盖我们的左右页边距。要做到这一点,我们使用命令\textbackslash fancyhfoffset[]\{\},并将其设置为等于我们的左右页边距,在本例中为 2.3 厘米,从而将页眉和页脚延伸到文档的边缘这里,还调用了软件包xcolor和graphicx,以便进行颜色选择和图表支持。
	           
	           要想在页面顶部加入一个全彩的矩形横幅,我们可以使用 \textbackslash fancyhead\{\}命令,并在其中心设置一个方框。要获得\LaTeX{} 的方框,可以使用命令\textbackslash fbox\{\},但如果我们想要一个彩色的方框,我们可以使用 \textbackslash fcolorbox\{frame color\}\{box background color\}\{...\}。
	                   
	       \subsubsection{使用 listings 进行高亮代码}
	           \begin{verbatim}
	               #include<stdio.h>
	               int main(){
	                   hello, world!;
	               }
	               return 0;
	               /* \begin{verbatim} ... */
	           \end{verbatim}
	           %[firstline=2, lastline =5] 只排版2-5行的代码
	           \begin{lstlisting}[
	               language=Python,
	               caption = 这是python代码,
	               ]
                    import numpy as np
    
                    def incmatrix(genl1,genl2):
                        m = len(genl1)
                        n = len(genl2)
                        M = None #to become the incidence matrix
                        VT = np.zeros((n*m,1), int)  #dummy variable
                        return M
                \end{lstlisting}
                \begin{lstlisting}[language=C, caption=C]
                    #include<stdio.h>
	               int main(){
	                   printf("hello, world!");//测试一下中文注释
	                   printf("hello, world!");//测试一下中文注释
	               }
	               return 0; 
                \end{lstlisting}

	       \subsubsection{画图}
	           Tikzit是一个简单的GUI工具,它允许绘制基本的图形和图表。它生成的Tikz/PGF输出代码可以直接嵌入在\LaTeX 代码中,可以在https://tikzit.github.io/下载。
	           
	           对于特别是数学绘图、草图、几何图形和其他图表,https://www.mathcha.io/ 提供了一个易于使用的图形环境,可以很容易地导出到\LaTeX 代码。
	           
	           为了绘制更复杂的图形,Inkscape是一个开放源码的矢量图形编辑器,它允许导出到PDF,可以让figures中的任何文本与主文件共享相同的字体。它可以在https://inkscape.org/下载。
	\section{浮动对象}
	
	\bibliography{refer}
\end{document}